\documentclass[12pt]{report}

% --- Essential Packages ---
\usepackage[utf8]{inputenc}
\usepackage[T1]{fontenc}
\usepackage[a4paper, margin=1in]{geometry}
\usepackage{graphicx} 
\usepackage[style=apa, backend=biber]{biblatex} % APA 7th Edition
\usepackage{setspace} % For line spacing
\usepackage{parskip}  % For spacing between paragraphs
\usepackage{times}    % Standard serif font for theses
\usepackage{hyperref} % For clickable Table of Contents

\addbibresource{references.bib}

% --- Global Formatting ---
\onehalfspacing      % Standard 1.5 spacing
\graphicspath{{./images/}}

\begin{document}

% --- 1. TITLE PAGE ---
\begin{titlepage}
    \begin{center}
        \vspace*{0.5cm}
        \includegraphics[width=0.4\textwidth]{strathmore_logo.png} \\
        \vspace{1cm}
        
        \textbf{\large STRATHMORE UNIVERSITY}\\
        \textbf{\large SCHOOL OF COMPUTING AND ENGINEERING SCIENCES}
        
        \vfill
        
        \textbf{\Large A PREDICTIVE ANALYTICS PLATFORM FOR STUDENT SUCCESS:}\\
        \vspace{0.5cm}
        \textbf{\large ML-BASED EARLY DETECTION OF AT-RISK STUDENTS IN HIGHER EDUCATION}
        
        \vfill
        
        BY\\
        \vspace{0.5cm}
        \textbf{\large KATHEMBO TSONGO DIEUDONNE}\\
        \textbf{112721}
        
        \vfill
        
        \textit{A Thesis Submitted in Partial Fulfillment of the Requirements for the Award of the Degree of Master of Science in Information Technology}
        
        \vfill
        
        NAIROBI, KENYA\\
        May, 2026
    \end{center}
\end{titlepage}

% --- 2. PRELIMINARIES (Roman Numerals) ---
\pagenumbering{roman}

\section*{Declaration and Approval}
I, the undersigned, declare that this thesis is my original work and has not been submitted for a degree in any other university.

\vspace{1.5cm}
\begin{tabular}{p{7cm} p{7cm}}
    Signature: \hrulefill & Date: \hrulefill \\
    \textbf{Kathembo Tsongo Dieudonne} & \\
\end{tabular}

\vspace{1.5cm}
This thesis has been submitted for examination with my approval as the University Supervisor.

\vspace{1.5cm}
\begin{tabular}{p{7cm} p{7cm}}
    Signature: \hrulefill & Date: \hrulefill \\
    \textbf{Dr. [Supervisor Name]} & \\
    Strathmore University & \\
\end{tabular}

\newpage
\tableofcontents
\newpage

% --- 3. MAIN BODY (Arabic Numerals) ---
\pagenumbering{arabic}

\chapter{Introduction}
\section{Background of the Study}

Student retention and academic success represent critical challenges confronting higher education institutions worldwide. According to \textcite{NCES2024}, undergraduate enrollment in degree-granting institutions decreased by 13\% between 2012 and 2022, while 23.3\% of undergraduate students leave universities annually. Six-year graduation rates reveal substantial variation: 63\% at public universities, 68\% at private nonprofit institutions, and only 29\% at private for-profit institutions. This persistent attrition represents significant personal and economic costs for individual students, alongside substantial challenges for institutional sustainability and mission fulfillment \parencite{Element4512024}.

Traditional student support approaches have been largely reactive, addressing academic difficulties only after failure occurs. However, the advent of artificial intelligence and machine learning has created unprecedented opportunities for proactive intervention through predictive analytics. Contemporary platforms now analyze hundreds of data points to forecast student outcomes with remarkable accuracy \parencite{Mapademics2025}. Georgia State University's pioneering implementation exemplifies this transformation, tracking over 800 risk factors for more than 40,000 students daily and generating 90,000 targeted interventions annually \parencite{GSU2024}. Machine learning algorithms can identify patterns that precede student disengagement with accuracy levels reaching 88--92\% \parencite{Ahmed2024, ScientificReports2025}, enabling early identification weeks or months before traditional indicators would surface.

In the Kenyan context, higher education institutions face distinctive challenges that intensify retention concerns. Public universities have experienced dramatic enrollment expansion while institutional resources have struggled to keep pace \parencite{Musasia2025}. Despite improvements in postgraduate retention from 75\% in 2015 to 87\% in 2024, substantial attrition remains \parencite{AJER2025}. Kenyan institutions confront severe funding constraints, with government capitation covering only 57\% of students against a target of 80\% \parencite{KIPPRA2024}. Financial pressures have intensified dramatically, with public universities accumulating pending bills of Ksh 62 billion as of February 2022 \parencite{DailyNation2024}, compounded by infrastructure deficits and severe staffing shortages where some lecturers teach triple their contracted student numbers \parencite{Visualdo2024}. In this resource-constrained environment, automated systems to identify and support at-risk students become essential for institutional survival and mission fulfillment.

Strathmore University operates within this challenging landscape while maintaining distinct academic standards, including a 67\% attendance requirement for examination eligibility and a 2.0 GPA threshold for good academic standing. The university collects extensive student data through Student Information Systems (SIS), Learning Management Systems (LMS), and attendance tracking mechanisms. However, these systems operate independently without integrated analytics capabilities, preventing transformation of existing data into actionable early warnings. Recent research demonstrates that ensemble methods such as Random Forest and XGBoost can achieve prediction accuracies exceeding 88\% when properly trained on comprehensive educational datasets \parencite{Turkmenbayev2025, ScientificReports2025b}. Yet successful implementation requires explainable models that provide clear insights into why particular students are flagged as at-risk and what specific interventions might prove most effective \parencite{FrontiersEducation2025}. This study addresses this gap by developing a machine learning-based predictive analytics platform tailored to Strathmore University's context, transforming reactive student support into proactive, data-driven intervention.

\section{Statement of the Problem}

Strathmore University faces significant challenges in identifying and supporting at-risk students before academic failure occurs. The university operates three comprehensive data systems that capture student demographics, academic performance, engagement metrics, and class participation. However, these systems operate independently without integrated analytics capabilities, preventing transformation of this data into actionable early warnings. Consequently, at-risk students are identified only after grades decline, attendance falls below the mandatory 67\% threshold, or dropout occurs---when intervention opportunities have largely passed. Academic advisors and mentors lack data-driven tools to prioritize limited resources, resulting in reactive rather than proactive intervention strategies.

This reactive approach creates multiple institutional challenges. Students who might have succeeded with timely support instead face academic probation, course repetition, extended time-to-graduation, or institutional departure. Faculty and administrators cannot identify systemic patterns in student risk, limiting evidence-based improvements. In Kenya's resource-constrained environment where government capitation covers only 57\% of students \parencite{KIPPRA2024}, each preventable failure represents both a personal setback and significant institutional loss.

Research demonstrates that machine learning systems achieve 85--92\% accuracy in identifying at-risk students \parencite{Ahmed2024, ScientificReports2025}, with proven implementations like Georgia State University's generating 90,000 targeted interventions annually \parencite{GSU2024}. Strathmore University currently lacks such capabilities despite possessing the necessary data infrastructure. This study develops a machine learning-based predictive analytics platform integrating existing institutional data to predict dropout risk, course failure risk, and program delay risk, enabling proactive intervention through role-based dashboards for administrators and mentors.

\section{Research Objectives}

\subsection{Main Objective}
To design, develop, and evaluate a machine learning-based predictive analytics platform for early identification of at-risk students at Strathmore University.

\subsection{Specific Objectives}
\begin{enumerate}
    \item To integrate student data from institutional systems (SIS, LMS, and attendance tracking) into a unified analytics repository.
    \item To develop machine learning models that predict dropout risk, course failure risk, and program delay risk.
    \item To design role-based dashboards for university administrators, school administrators, and academic mentors with actionable risk predictions.
    \item To evaluate model accuracy and predictive performance using historical student data and standard evaluation metrics.
    \item To assess system usability and institutional impact through stakeholder evaluation.
\end{enumerate}

\section{Research Questions}

\subsection{Main Research Question}
How can machine learning be effectively applied to institutional student data to enable early identification of at-risk students at Strathmore University?

\subsection{Specific Research Questions}
\begin{enumerate}
    \item What student data features from SIS, LMS, and attendance systems are most predictive of dropout risk, course failure risk, and program delay risk?
    \item Which machine learning algorithms provide the most accurate predictions for each of the three risk categories?
    \item How can risk predictions be effectively presented to different stakeholder groups to support decision-making?
    \item What level of prediction accuracy can be achieved using historical student data from Strathmore University?
    \item How do university stakeholders perceive the usability and potential institutional impact of the predictive analytics platform?
\end{enumerate}

\section{Scope and Delimitations of the Study}

\subsection{Scope}
This research focuses specifically on developing and evaluating a predictive analytics platform for student success at Strathmore University, Kenya. The study encompasses undergraduate student data from all five schools: the School of Computing and Engineering Sciences (SCES), Strathmore Business School (SBS), the School of Tourism and Hospitality (STH), Strathmore Law School (SLS), and the School of Humanities and Social Sciences (SHSS). The research utilizes data from approximately 440 undergraduate students, providing sufficient statistical power for reliable machine learning model development.

The technical scope includes integration of data from three primary institutional systems: the Student Information System (academic records), the Learning Management System (engagement metrics), and the attendance tracking system (class participation). The study develops and evaluates three distinct machine learning prediction models targeting dropout risk, course failure risk, and program delay risk, employing ensemble algorithms including Random Forest, Gradient Boosting, and XGBoost. The platform design encompasses role-based dashboards serving university administrators, school administrators, and academic mentors. System effectiveness is evaluated through quantitative metrics (accuracy, precision, recall, F1-score, ROC-AUC) and qualitative stakeholder assessment.

\subsection{Delimitations}
Several deliberate delimitations define the boundaries of this research. First, the study focuses exclusively on academic risk factors measurable through institutional data systems, excluding predictions of mental health concerns, family circumstances, or socioeconomic factors that fall outside institutional data collection. Second, the research limits its focus to undergraduate students, excluding postgraduate and continuing education populations whose risk patterns differ substantially. Third, the study does not develop course-specific recommendation systems or personalized learning pathways; the platform identifies at-risk students and suggests broad intervention categories without prescribing specific courses or detailed academic plans.

Fourth, the platform is designed as a web-based application accessible through standard browsers, deliberately excluding mobile application development to prioritize cross-platform accessibility and reduce development complexity. Fifth, the research does not integrate financial aid prediction or scholarship recommendation capabilities, despite the documented importance of financial factors in Kenyan student retention. Finally, while the research develops a fully functional prototype system, full production deployment across the entire Strathmore student population represents future work beyond the thesis scope, requiring additional institutional approvals, infrastructure provisioning, and ongoing technical support arrangements.

\section{Significance of the Study}

\subsection{Theoretical Contribution}
This study advances scholarly understanding of educational data mining in resource-constrained environments. While existing literature predominantly focuses on well-resourced Western institutions \parencite{Turkmenbayev2025, ScientificReports2025}, this research demonstrates the applicability of advanced predictive analytics where data infrastructure may be limited. The study operationalizes Tinto's Student Integration Model by mapping theoretical constructs to quantifiable data elements, providing empirical validation of retention frameworks within an African context.

\subsection{Practical Contribution}

\subsubsection{For Institutions}
The platform provides Strathmore University and similar institutions with tools to transform reactive support into proactive intervention. Data-driven resource allocation directs limited counseling and mentoring resources toward students most likely to benefit. The platform generates actionable insights about program-level risk patterns, informing curriculum development and policy adjustments \parencite{ModernCampus2025}.

\subsubsection{For Educators}
The platform equips academic advisors with evidence-based insights into student risk factors, moving beyond intuition to data-informed strategies. Role-based dashboards highlight specific concerns---attendance patterns, engagement deficits, assignment completion---enabling targeted interventions. Early identification creates opportunities for proactive outreach before problems become insurmountable.

\subsubsection{For Students}
While not direct users, students are ultimate beneficiaries. Timely interventions can prevent academic failures, reduce program delays, and increase graduation likelihood. The system ensures that institutional policies like the 67\% attendance requirement are supported by compliance mechanisms rather than serving merely as punitive measures.

\subsection{Policy Contribution}
This research demonstrates a cost-effective approach to addressing retention challenges in African universities, providing a replicable model. Given Kenya's higher education challenges \parencite{KIPPRA2024, DailyNation2024}, the study illustrates how strategic application of artificial intelligence can amplify limited resources without proportional expenditure increases. By documenting implementation challenges and lessons learned, the study creates knowledge that can accelerate adoption across Kenyan and African institutions \parencite{KenyaEducation2024}.
% --- 4. BIBLIOGRAPHY ---
\newpage
\printbibliography[title={References}]

\end{document}