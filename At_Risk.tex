\documentclass[12pt]{report}

% --- Essential Packages ---
\usepackage[utf8]{inputenc}
\usepackage[T1]{fontenc}
\usepackage[a4paper, margin=1in]{geometry}
\usepackage{graphicx} 
\usepackage[style=apa, backend=biber]{biblatex} % APA 7th Edition
\usepackage{setspace} % For line spacing
\usepackage{parskip}  % For spacing between paragraphs
\usepackage{times}    % Standard serif font for theses
\usepackage{hyperref} % For clickable Table of Contents

\addbibresource{references.bib}

% --- Global Formatting ---
\onehalfspacing      % Standard 1.5 spacing
\graphicspath{{./images/}}

\begin{document}

% --- 1. TITLE PAGE ---
\begin{titlepage}
    \begin{center}
        \vspace*{0.5cm}
        \includegraphics[width=0.4\textwidth]{strathmore_logo.png} \\
        \vspace{1cm}
        
        \textbf{\large STRATHMORE UNIVERSITY}\\
        \textbf{\large SCHOOL OF COMPUTING AND ENGINEERING SCIENCES}
        
        \vfill
        
        \textbf{\Large A PREDICTIVE ANALYTICS PLATFORM FOR STUDENT SUCCESS:}\\
        \vspace{0.5cm}
        \textbf{\large ML-BASED EARLY DETECTION OF AT-RISK STUDENTS IN HIGHER EDUCATION}
        
        \vfill
        
        BY\\
        \vspace{0.5cm}
        \textbf{\large KATHEMBO TSONGO DIEUDONNE}\\
        \textbf{112721}
        
        \vfill
        
        \textit{A Thesis Submitted in Partial Fulfillment of the Requirements for the Award of the Degree of Master of Science in Information Technology}
        
        \vfill
        
        NAIROBI, KENYA\\
        May, 2026
    \end{center}
\end{titlepage}

% --- 2. PRELIMINARIES (Roman Numerals) ---
\pagenumbering{roman}

\section*{Declaration and Approval}
I, the undersigned, declare that this thesis is my original work and has not been submitted for a degree in any other university.

\vspace{1.5cm}
\begin{tabular}{p{7cm} p{7cm}}
    Signature: \hrulefill & Date: \hrulefill \\
    \textbf{Kathembo Tsongo Dieudonne} & \\
\end{tabular}

\vspace{1.5cm}
This thesis has been submitted for examination with my approval as the University Supervisor.

\vspace{1.5cm}
\begin{tabular}{p{7cm} p{7cm}}
    Signature: \hrulefill & Date: \hrulefill \\
    \textbf{Dr. [Supervisor Name]} & \\
    Strathmore University & \\
\end{tabular}

\newpage
\tableofcontents
\newpage

% --- 3. MAIN BODY (Arabic Numerals) ---
\pagenumbering{arabic}

\chapter{Introduction}

\section{Background of the Study}
Student retention and academic success represent critical challenges confronting 
higher education institutions worldwide. Recent data indicates that while persistence 
rates have rebounded to pre-pandemic levels, reaching 76.5\% nationally in 2024, nearly 
one in four undergraduate students still fail to return after their first year \parencite{ModernCampus2025, Hanover2025}. 
This persistent attrition represents not only significant personal and economic costs for individual students but 
also substantial challenges for institutional sustainability \parencite{Element4512024}.

The advent of artificial intelligence and machine learning (ML) has revolutionized approaches to student success analytics. 
Contemporary platforms have evolved into comprehensive ecosystem monitors that analyze hundreds of data points to forecast 
outcomes \parencite{Mapademics2025}. Georgia State University's implementation tracks over 800 risk factors daily, 
generating 90,000 targeted interventions annually \parencite{Mapademics2025}. ML algorithms now identify patterns that
precede student disengagement with accuracy levels reaching 88--92\% \parencite{Ahmed2024, ScientificReports2025}.

In the African context, Kenyan institutions face distinctive challenges. While postgraduate retention improved 
to 87\% in 2024, significant attrition remains \parencite{AJER2025}. Universities confront multifaceted hurdles 
including inadequate funding (covering only 57\% of students), infrastructure deficits, and staffing 
shortages \parencite{KIPPRA2024, Visualdo2024, Musasia2025}.

Strathmore University, despite collecting extensive data through SIS and LMS platforms, lacks integrated, 
automated tools to proactively identify students at risk. The current reactive approach—addressing difficulties 
only after grades decline—represents a missed opportunity. Recent research demonstrates that ensemble methods like
 Random Forest and XGBoost can achieve prediction accuracies 
 exceeding 88\% \parencite{Turkmenbayev2025, ScientificReports2025}.
  Successful implementation requires transparent, interpretable models to provide clear insights for
   interventions \parencite{Frontiers2025}.

\section{Statement of the Problem}
Despite the availability of student data, many institutions lack automated systems to 
detect early warning signs of academic struggle. This leads to high attrition rates that 
could be mitigated through timely intervention.

\section{Purpose of the Study}
The purpose of this research is to develop a predictive analytics platform that leverages student demographic and engagement data to improve retention rates.

\section{Research Objectives}
\begin{itemize}
    \item To identify the key indicators of academic risk in higher education.
    \item To develop a machine learning model for predicting student performance.
    \item To evaluate the accuracy and reliability of the predictive platform.
\end{itemize}

\section{Research Questions}
\begin{enumerate}
    \item What features in student data are most predictive of academic failure?
    \item How accurately can a Random Forest or Neural Network model detect at-risk students?
\end{enumerate}

\section{Significance of the Study}
This study is significant for university administrators, educators, and students, as it provides a data-driven approach to academic support.

% --- 4. BIBLIOGRAPHY ---
\newpage
\printbibliography[title={References}]

\end{document}